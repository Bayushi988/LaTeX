\documentclass[10pt]{beamer}

\usepackage[utf8]{inputenc}
\usepackage[T1]{fontenc}
\usepackage[french]{babel}
\usepackage{tabularx}
\usepackage{tkz-tab}
\usepackage{caption}
\usepackage{amsthm,amsmath,amssymb}
\usepackage{mathrsfs}
\usepackage{pstricks,pstricks-add,pst-plot,pst-tree,pst-node}
\usepackage{tikz,tkz-tab}
\usepackage{bm}

\DecimalMathComma
\frenchbsetup{StandardLists=true}
\usepackage[autolanguage,np]{numprint}
\usepackage{colortbl}

\newtheorem{theo}{Théorème}
\newtheorem{prop}{Proposition}
\theoremstyle{definition}
\newtheorem{defin}{Définition}
\newtheorem{rmq}{Remarque}
\theoremstyle{example}
\newtheorem{exemple}{Exemple}
\newtheorem{exemples}{Exemples}

\newcommand{\N}{\ensuremath{\mathbb{N}}}
\newcommand{\Z}{\ensuremath{\mathbb{Z}}}
\newcommand{\D}{\ensuremath{\mathbb{D}}}
\newcommand{\Q}{\ensuremath{\mathbb{Q}}}
\newcommand{\R}{\ensuremath{\mathbb{R}}}
\newcommand{\C}{\ensuremath{\mathbb{C}}}
\newcommand{\OIJ}{\ensuremath{\left(\mathrm{O},~\mathrm{I},~\mathrm{J}\right)}}
\newcommand{\Oij}{\ensuremath{\left(\mathrm{O}~;~\vec{\imath},~\vec{\jmath}\right)}}
\newcommand{\e}[1]{\mathrm{e}^{#1}}
\newcommand{\Card}[1]{\ensuremath{\mathrm{Card}\left(#1\right)}}
\newcommand{\intervFF}[2]{\ensuremath{\left[#1~;~#2\right]}}
\newcommand{\intervFO}[2]{\ensuremath{\left[#1~;~#2\right[}}
\newcommand{\intervOF}[2]{\ensuremath{\left]#1~;~#2\right]}}
\newcommand{\intervOO}[2]{\ensuremath{\left]#1~;~#2\right[}}


\usetheme{Luebeck}
\usefonttheme{serif}

\title[Chapitre 2]{{\small{Spécialité mathématiques première}}\\Chapitre 2 : Trigonométrie}
\author{S. Dibos}
\institute{Lycée Dick Ukeiwë}
\date{Année 2024}

\AtBeginSection[]
{
\begin{frame}{Plan}
\tableofcontents[currentsection]
\end{frame}
}

\begin{document}

\begin{frame}
	\titlepage
\end{frame}

\section{Enroulement de la droite des réels}
\subsection{Cercle trigonométrique}

\begin{frame}{Cercle trigonométrique}
\transdissolve[duration=2]
\begin{overprint}
\only<1->{%
\only<1>{\Oij{} désigne un repère orthonormal du plan.}
\begin{defin}
On appelle \alert<1>{\textbf{cercle trigonométrique}}, le cercle de centre O, de rayon 1 et orienté dans le sens positif (de $\vec{\imath}$ vers $\vec{\jmath}$).
\end{defin}}
\only<2->{%
\begin{center}
\psset{unit=1.375cm,algebraic=true,dotstyle=+,dotsize=5pt 0, linewidth=1.1pt,arrowsize=4pt 1,arrowinset=.25}
\begin{pspicture*}(-1.6,-1.6)(1.6,1.6)
\psgrid[gridlabels=0,subgriddiv=5,gridcolor=cyan,subgridcolor=cyan](0,0)(-1.6,-1.6)(1.6,1.6)
\psaxes[labels=none,ticksize=2pt 0,subticks=2,arrowsize=3pt 1]{->}(0,0)(-1.5,-1.5)(1.5,1.5)
\psline[linewidth=1.25pt]{<->}(0,1)(0,0)(1,0)
\pscircle[linecolor=blue](0,0){1}
\psarc[linecolor=red]{->}(0,0){1.5}{35}{70}
\begin{scriptsize}
\uput[-135](0,0){O} \uput[-90](0.5,0){$\vec{\imath}$} \uput[180](0,0.5){$\vec{\jmath}$} \uput[120](-0.5,0.866){\blue{\small{$\Gamma$}}} \uput[45](0.913,1.19){\red{\large{\textbf{+}}}}
\end{scriptsize}
\end{pspicture*}
\captionof{figure}{Cercle trigonométrique}
\end{center}}
\end{overprint}
\end{frame}

\subsection{Enroulement de la droite réelle}

\begin{frame}{Enroulement de la droite réelle}
\transdissolve[duration=2]
\begin{overprint}
\begin{columns}
\begin{column}{.375\textwidth}
\only<1->{%
On considère la droite $\left(d\right)$ tangente au cercle $\Gamma$ en I et munie du repère $\left(\mathrm{I},~\vec{\jmath}\right)$. À tout point P d'abscisse $x$ de la droite $\left(d\right)$, on associe un unique point M sur le cercle trigonométrique. Pour cela, on \og enroule\fg{} le segment $\left[\mathrm{IP}\right]$ autour de $\Gamma$.

\onslide<2>{La position de M sur le cercle trigonométrique est donc associée à $x$. M est appelé \textbf{l'image} du réel $x$ sur le cercle trigonométrique.}}
\end{column}
\begin{column}{.525\textwidth}
\only<1->{%
\begin{center}
\psset{unit=1.25cm,algebraic=true,dimen=middle,dotsize=5pt 0, linewidth=1.1pt,arrowsize=4pt 1,arrowinset=.25}
\begin{pspicture*}(-1.4,-1.5)(1.8,3)
\psgrid[gridlabels=0,subgriddiv=5,gridcolor=cyan,subgridcolor=cyan](0,0)(-1.4,-1.4)(1.8,3)
\psaxes[labels=none,ticksize=4pt 0,subticks=2,arrowsize=3pt 1]{->}(0,0)(-1.25,-1.25)(1.75,2.75)
\psline[linewidth=1.75pt]{<->}(0,1)(0,0)(1,0)
\pscircle[linecolor=blue](0,0){1}
\psline(1,-1.25)(1,2.75)
\psdots(1,0)
\psline[linewidth=1.75pt]{->}(1,0)(1,1)
\onslide<2->{\psdots[dotstyle=*,linecolor=red](1,2.18)(-0.572,0.82)
\psarc[linecolor=red]{->}(1.675,-0.1){2.4}{110}{155}}
\begin{scriptsize}
\uput[-135](0,0){O} \uput[-90](0.5,0){$\vec{\imath}$} \uput[-180](0,0.5){$\vec{\jmath}$} \uput{0.2}[30](0.05,0.985){\blue{$\Gamma$}} \uput[-45](1,0){I} \onslide<2>{\uput[120](1,2.18){\red{P}} \uput[0](1,2.18){\red{$x$}} \uput[180](-0.572,0.82){\red{M}}}
\end{scriptsize}
\end{pspicture*}
\end{center}}
\end{column}
\end{columns}
\end{overprint}
\end{frame}

\subsection{Le radian}

\begin{frame}

\end{frame}

\section{Repérage d'un point sur le cercle trigonométrique}
\subsection{Cosinus et sinus d'un réel $x$}

\begin{frame}

\end{frame}

\subsection{Propriétés du cosinus et du sinus}

\begin{frame}

\end{frame}

\subsection{Valeurs remarquables du cosinus et du sinus}

\begin{frame}

\end{frame}

\end{document}